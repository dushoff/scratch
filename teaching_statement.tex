\documentclass[12pt]{article}
\usepackage[top=2cm, bottom=2cm, left=3.5cm, right=3.5cm]{geometry}

\begin{document}

\section*{Teaching statement -- Jonathan Dushoff }

Ever since I can remember, I have enjoyed teaching in small, informal settings.  Teaching large lectures is rather different; while I enjoy interacting with students, and the feeling of accomplishment, I see teaching a large lecture as a heavy responsibility, to be approached with a great deal of preparation.

At Princeton University, I taught in a variety of different settings, not always with as much success as I would have liked.  In retrospect, I sometimes focused too much on presenting material in an accurate and logical fashion, and not enough on repeating key ideas in different ways, and on interacting with the students.

Since arriving at McMaster, I've taught three different large courses: two different versions of introductory biology (both focusing on Ecology and Evolution), and a third-year population biology course.  The introductory courses had around 1500 students, and are taught in four separate sections.  The experience has been very different from all of my previous teaching experiences, which were in my sub-specialties of mathematical and statistical biology.  
My section of the course is very interactive for a large course.
I never go through a class period without stopping at least once and asking the students to think about a question (either alone or in small groups), and then discussing the responses.
Several observers have commented on my ability to extract both responses and questions from such large lecture sections.
I go through all of the terminology and definitions that I plan to require, but the bulk of my lecture time is spent on a handful of conceptual issues from each chapter.
I strive to engage the students in thinking about these questions, and I approach the key ones from several different angles.
I try to minimize the amount of material that students need to memorize, explicitly identify every term that I want them to learn, and tell them that they are responsible for concepts from the book, but details only from the lecture notes.

My fundamental philosophy of first-year teaching is to focus on conceptual, and to use the material largely as a framework within which to talk about basic concepts of logic, inference and experiment.  A basic theme that runs through my lectures is that concepts are fundamentally simple, but that thinking clearly in the abstract is hard.  Since I teach evolutionary biology, I put this in the context of a very concrete world in which we have evolved.  I encourage students to use a variety of techniques to make the stories and ideas of science familiar to them, since I believe that this is a very helpful aid to clear thinking in many cases.  My main goal for students in the introductory course is that they learn a few basic facts, become familiar with basic scientific and logical thinking, and develop tools for learning how to think about new systems.

I did better than I expected with the new experience of a very large class in a very different subject area, both in terms of my own feeling about what is covered in lecture and the lecture-hall atmosphere, and the student evaluations.  My interactions with students are positive, I have much better than average quality ratings for an introductory course (but only average ``easiness'' ratings), and have been nominated for a teaching award.

In Population Biology (a third-year course), I take a very conceptual approach, focusing on the very basics of population growth, age structure, competition and trophic interactions.  I taught only the simplest mathematical concepts, but tried to probe them in a deep way.  My course experience and reviews after two years are good, but not as positive as I have had with Introductory Biology.  I am working to organize the course more strongly, and to find ways to reach the students in different ways without compromising my goal of making them engage with the theoretical and mathematical concepts that I feel are critical to understanding this subject.  I have taught the course twice now, and hope to get it right the third time around.

I have been surprised that Population Biology has been more of a challenge for me than Introductory Biology was, but I have embraced this challenge. When I received a salary award that would have allowed me to teach only Introductory Biology (or to negotiate to avoid large classes altogether), I chose to stick with Population Biology.  I have worked to streamline the course and to focus on areas that give the students conceptual difficulty. I also continue to strive to make the third-year course more open-ended and interactive, challenging the students to work out ideas for themselves without overwhelming them.  As with the first-year course, I try to make the classroom interactive and keep the students engaged; in my experience, this is more difficult with third-year students.

Additionally, I have committed a lot of time to building the ICI3D program. This is a program designed to help junior researchers learn how to conduct integrative research in infectious-disease dynamics. We sponsor an annual two-week workshop in Cape Town, an annual one-week clinic in Florida, and also sponsor trans-continental research visitors. ICI3D grew from collaborative teaching efforts that started around the time I arrived at McMaster, and I have been a member of their core faculty from the beginning. Working on ICI3D has been very demanding, and very rewarding: my close interactions with the other faculty have helped me greatly expand my range of pedagogical techniques.

\end{document}


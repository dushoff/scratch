\documentclass[12pt,a4paper]{article}

\usepackage[top=2cm, bottom=2cm, left=3.5cm, right=3.5cm]{geometry}

\begin{document}

\newcommand{\one}[1]{\vspace{\baselineskip}\hspace{-\parindent}{\large \bf #1}}
\newcommand{\two}[1]{\vspace{\baselineskip}\hspace{-\parindent}{\it #1}}
\newcommand{\square}{\vspace{1.5ex}\noindent}

\thispagestyle{empty}

\one{Research Statement -- Jonathan Dushoff}

\square My primary research interest is using mathematical and computational methods to better understand the evolution, spread and health burden of infectious diseases of humans.  My work combines a variety of approaches, including population biology, population genetics, genomics, immunological modeling and statistical analysis of epidemiological data.

I am interested in infectious disease for both practical and theoretical reasons.  The burden of infectious disease world-wide, particularly on poor countries and communities, is far larger than it should be, and some of my work is aimed at directly or indirectly influencing policy decisions and control strategies.  Additionally, I believe that infectious diseases provide excellent model systems for a variety of evolutionary and population biology questions.

My research uses a variety of mathematical and computational methods to link disease processes across scales, building from within-individual immunological interactions and mutation events, to between-individual infection patterns, to population-level patterns of disease spread and evolution.  My work is highly collaborative, and I have worked on a broad variety of projects with colleagues from disciplines from math, to physics, to ecology to molecular biology.  Mathematically, I am particularly interested in insights from analytic approximations to complex dynamical models, and in novel statistical methods that make use of computational patterns to relax traditional statistical assumptions. Scientifically, my current main foci of interest are influenza, rabies and HIV.

My interests in influenza range from molecular evolution, to classic epidemiology, to statistics.  In the last few years, a huge number of influenza full-genome sequences have become available.  This rich data set allows patterns of evolution to be studied in new ways, and demands new techniques for doing so.  For example, collaborators and I have found evidence for non-random trends in nucleotide composition as influenza viruses evolve, and evidence that patterns of substitution cannot be explained using the traditional paradigm of positively and negatively selected sites.  We are currently investigating mechanistic hypotheses for observed patterns of nucleotide change.  On the epidemiology front, collaborators and I have explored new ways of make use of influenza mortality time series, for example: by harvesting additional information from publicly available death records; developing statistical analysis methods robust to seasonal confounding; and currently by developing new techniques to infer morbidity series from mortality data, and to infer disease parameters from morbidity data.  Finally, collaborators and I continue to construct deterministic and stochastic dynamical models which illuminate the interpretation of incidence patterns, including patterns of influenza spread in the light of climatic data, covering both the 1918 and 2009 pandemics, as well as ``annual" flu epidemics from Canada.

I am fortunate to be associated with a very exciting research project, led by Katie Hampson, which has brought the serious public health problem of human mortality caused by canine rabies back into the focus of mainstream science.  I believe that the contributions that I made to this project in terms of dynamical modeling and statistical methods have been both practically important to unravelling patterns in the data, and also theoretically interesting.  In particular, collaborators and I developed new, permutation-based methods for robustly analyzing rabies incidence time series, created deterministic and stochastic dynamical models of rabies spread on different scales, and used a variety of statistical techniques to robustly analyze detailed rabies-transmission data. We are currently involved in evaluating progress in rabies elimination, with a recently published paper on worldwide rabies burden, and current projects on monitoring guidelines, and describing the recent dramatic decline in canine rabies in the Western hemisphere.

Recent progress in HIV treatment, and public health approaches, make this a very exciting time for HIV research: there is a real prospect that the epidemic will be brought under control. With lab members and other collaborators, I have engaged in a wide range of HIV research projects. We have investigated interactions between HIV spread and behaviour change, with particular emphasis on gendered concerns. We have also looked at routes of HIV spread: how effectively does HIV spread during the early, middle and late stages of infection, how much spread is due to coupled vs.\ uncoupled people; among coupled people, how much spread is within-couple vs.\ extra-couple. Now that WHO has recommended drug treatment for all HIV-positive people (a recommendation we have advocated for years), we are turning our attention to the ``treatment cascade'' -- issues related to outreach, testing, and linkage to care -- and also health outcomes. We have ongoing projects with collaborators in both Uganda and Tanzania. 

I had the opportunity to contribute to a variety of investigations about the recent West African Ebola outbreak, including issues involving safe burial, sub-clinical infection, and design of ethical vaccine trials. This work has also inspired me to reassess ideas about generation times and the ``strength and speed'' of epidemics, leading to a number of ongoing projects.

I plan to continue investigating pathogen biology across scales, from gene-sequence evolution and protein shape and function, through pathogenesis and immunology at the level of individual hosts, to community- and population-level patterns of transmission, pathogen evolution and disease burden.  My overarching goal is to use quantitative techniques to investigate basic and applied biological questions, and to bring this research to bear on practical questions relating to the control of infectious diseases of humans.

I am also actively pursuing ecological research. I am excited about my recently funded NSERC proposal on interactions between the spread of information through populations, and the spread of behaviours. I have collaborators who work on systems ranging from bacteria, to large non-human mammals, to more sociological systems, and am planning to use a variety of modeling techniques to explore similarities and differences in population-level patterns. I am also actively involved in efforts to characterize and compare patterns of diversity in ecological systems.

\end{document}
